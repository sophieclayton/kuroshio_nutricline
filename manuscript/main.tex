%%%%%%%%%%%%%%%%%%%%%%%%%%%%%%%%%%%%%%%%%%%%%%%%%%%%%%%%%%%%%%%%%%%%%%%%%%%%%%%%%%%%%%%%%%%%%%%%%%%%%%%%%%%%%%%%%%%%%%%%%%%%%%%%%%%%%%%%%%%%%%%%%%%%%%%%%%%
% This is just an example/guide for you to refer to when submitting manuscripts to Frontiers, it is not mandatory to use Frontiers .cls files nor frontiers.tex  %
% This will only generate the Manuscript, the final article will be typeset by Frontiers after acceptance.   
%                                              %
%                                                                                                                                                         %
% When submitting your files, remember to upload this *tex file, the pdf generated with it, the *bib file (if bibliography is not within the *tex) and all the figures.
%%%%%%%%%%%%%%%%%%%%%%%%%%%%%%%%%%%%%%%%%%%%%%%%%%%%%%%%%%%%%%%%%%%%%%%%%%%%%%%%%%%%%%%%%%%%%%%%%%%%%%%%%%%%%%%%%%%%%%%%%%%%%%%%%%%%%%%%%%%%%%%%%%%%%%%%%%%

%%% Version 3.4 Generated 2018/06/15 %%%
%%% You will need to have the following packages installed: datetime, fmtcount, etoolbox, fcprefix, which are normally inlcuded in WinEdt. %%%
%%% In http://www.ctan.org/ you can find the packages and how to install them, if necessary. %%%
%%%  NB logo1.jpg is required in the path in order to correctly compile front page header %%%

\documentclass[utf8]{frontiersSCNS} % for Science, Engineering and Humanities and Social Sciences articles
%\documentclass[utf8]{frontiersHLTH} % for Health articles
%\documentclass[utf8]{frontiersFPHY} % for Physics and Applied Mathematics and Statistics articles

%\setcitestyle{square} % for Physics and Applied Mathematics and Statistics articles
\usepackage{url,hyperref,lineno,microtype,subcaption}
\usepackage[onehalfspacing]{setspace}

\linenumbers


% Leave a blank line between paragraphs instead of using \\


\def\keyFont{\fontsize{8}{11}\helveticabold }
\def\firstAuthorLast{Clayton} %use et al only if is more than 1 author
\def\Authors{Sophie Clayton$^{1,*}$}
% Affiliations should be keyed to the author's name with superscript numbers and be listed as follows: Laboratory, Institute, Department, Organization, City, State abbreviation (USA, Canada, Australia), and Country (without detailed address information such as city zip codes or street names).
% If one of the authors has a change of address, list the new address below the correspondence details using a superscript symbol and use the same symbol to indicate the author in the author list.
\def\Address{$^{1}$Department of Ocean and Earth Sciences, Old Dominion University, Norfolk , VA, USA}
% The Corresponding Author should be marked with an asterisk
% Provide the exact contact address (this time including street name and city zip code) and email of the corresponding author
\def\corrAuthor{Sophie Clayton}

\def\corrEmail{sclayton@odu.edu}




\begin{document}
\onecolumn
\firstpage{1}

\title{Spatial and temporal variability in nutricline depth in the Kuroshio and Oyashio Extensions from BGC-Argo floats} 

\author[\firstAuthorLast]{\Authors} %This field will be automatically populated
\address{} %This field will be automatically populated
\correspondance{} %This field will be automatically populated

\maketitle


\begin{abstract}
The depth of the nutricline and the magnitude of vertical nutrient fluxes have a direct influence on primary productivity and the strength of the biological carbon pump. Western boundary currents have been shown to be hotspots for carbon cycling, largely due to their high levels of eddy kinetic energy superimposed over large latitudinal gradients in physical and biogeochemical properties. The presence of mesoscale eddies and submesoscale filaments drives variability in the mixed layer and the nutricline depth over a wide range of space and time scales. The Kuroshio and Oyashio have been shown to act as large-scale subsurface nutrient streams, supporting large lateral transports of nutrients within the upper thermocline, and these nutrient streams are characterized by persistent along-isopycnal positive nitrate anomalies. However the extent of the spatial and temporal variability of the nutricline depth in the north western Pacific is not well known, partly due to the relatively sparse number of nutrient profiles over the broader region. In this study, data from a large number of continuous nitrate profiles from BGC-Argo floats deployed in the western Pacific, spanning multiple years and seasons, are used to determine temporal and spatial variability in the nutricline depth and how they relate to physical dynamics and primary productivity estimated from remote sensing.


\section{}



\tiny
 \keyFont{ \section{Keywords:} western boundary currents, nitrate, nutricline, upper ocean processes, , primary production, Kuroshio} %All article types: you may provide up to 8 keywords; at least 5 are mandatory.
\end{abstract}

\section{Introduction}
Nutricline depth can be used as a proxy for nutrient availability as it scales with the mixed layer integrated nutrient supply (add in equation here to show how deeper nutricline results in decreased MLD averaged nutrient supply?)

Importance of nutricline depth in setting rates or primary production \citep{richardson2019vertical}

Kuroshio has variability in nutricline due to a range of processes \citep{clayton2021synoptic}

Previous study looking at large scale patterns in nitracline depth and shape \citep{omand2015shape}... not here is a lot more data around to take a more detailed look at spatiala dn temporal variability in the nitracline in the Kuroshio Extension region thanks to more BGC-Argo floats being deployed over the last few years...

Nutricline estimates from continuous profiles rather than discrete data points

Questions to be answered: 
- how does the nutricline depth and density level vary seasonally?
- does the mean depth of the nutricline shoal as you move downstream along the Kuroshio?

\section{Data and Methods}
\subsection{BGC-Argo float data}
List float numbers, sensors included and duration of deployment
- this is not looking as QA/QC'd files are ugly

\begin{table}[]
\caption{Float information for BGC-Argo floats used in this study.}
\begin{tabular}{|c|c|l|l|l|c|}
\hline
WMO number & DAC & BGC-sensors &Start date  &End date & \# profiles\\
\hline
2902754 &  &  &  &  & \\
2902755 &  &  &  &  & \\
2903329 &  &  &  &  & \\
2903330 & & & & & \\
2903394 & & & & & \\
2903396 & & & & & \\
2903672 & & & & & \\
5904034 & & & & & \\
5904035 & & & & & \\
\hline
\end{tabular}
\label{floats}
\end{table}
Float data for the floats shown in \ref{floats} were downloaded and can be accessed from  the Argo GDAC \citep{argo2022}


QA/QC on float data \citep{maurer2021delayed}

Figure 1. Data density in time/space (map showing float trajectories and number of profiles in 1x1 degree boxes, histogram showing prfolies by month of year)

\subsection{Nutricline estimation}
Which criterion to use? 

% \subsection{Nitrate flux estimates}
% Dissipation rate estimation using Thorpe scale method (refer to Whalen et al. 2012 and Han et al., 2021)


\section{Results}
\subsection{Nutricline depths}
compare to WOA-deried nutriclines... is there a consistent trend in the difference between argo nutricline and WOA nutricline? (not sure this makes sense necessarily)

\subsection{Relationship between nutricline depth and SSH}

\subsection{Relationship between nutricline depth and satellite NPP}


\section{Discussion}


\section{Conclusions}


\section*{Conflict of Interest Statement}
%All financial, commercial or other relationships that might be perceived by the academic community as representing a potential conflict of interest must be disclosed. If no such relationship exists, authors will be asked to confirm the following statement: 

The author declares that the research was conducted in the absence of any commercial or financial relationships that could be construed as a potential conflict of interest.

\section*{Author Contributions}
SC conceived the study, performed the analyses and prepared the manuscript.

\section*{Funding}


\section*{Acknowledgments}
The BGC-Argo float data used in this study were collected and made freely available by the International Argo Program and the national programs that contribute to it (\href{http://www.argo.ucsd.edu}{http://www.argo.ucsd.edu},  \href{http://argo.jcommops.org}{http://argo.jcommops.org}). The Argo Program is part of the Global Ocean Observing System. The data used in this study can be accessed using the following DOI: \href{http://doi.org/10.17882/42182#91554}{http://doi.org/10.17882/42182#91554}

\bibliographystyle{frontiersinSCNS_ENG_HUMS} 

\bibliography{kuro}

%%% Make sure to upload the bib file along with the tex file and PDF
%%% Please see the test.bib file for some examples of references

\section*{Figure captions}

% only 2 figs/tables allowed

% \begin{figure}[h!]
% \begin{center}
% \includegraphics[width=10cm]{logo1}% This is a *.eps file
% \end{center}
% \caption{ Enter the caption for your figure here.  Repeat as  necessary for each of your figures}\label{fig:1}
% \end{figure}


% \begin{figure}[h!]
% \begin{center}
% \includegraphics[width=15cm]{logos}
% \end{center}
% \caption{This is a figure with sub figures, \textbf{(A)} is one logo, \textbf{(B)} is a different logo.}\label{fig:2}
% \end{figure}

%%% If you are submitting a figure with subfigures please combine these into one image file with part labels integrated.
%%% If you don't add the figures in the LaTeX files, please upload them when submitting the article.
%%% Frontiers will add the figures at the end of the provisional pdf automatically
%%% The use of LaTeX coding to draw Diagrams/Figures/Structures should be avoided. They should be external callouts including graphics.

\end{document}
